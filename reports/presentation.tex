%%%%%%%%%%%%%%%%%%%%%%%%%%%%%%%%%%%%%%%%%
% Beamer Presentation
% LaTeX Template
% Version 1.0 (10/11/12)
%
% This template has been downloaded from:
% http://www.LaTeXTemplates.com
%
% License:
% CC BY-NC-SA 3.0 (http://creativecommons.org/licenses/by-nc-sa/3.0/)
%
%%%%%%%%%%%%%%%%%%%%%%%%%%%%%%%%%%%%%%%%%

%----------------------------------------------------------------------------------------
%	PACKAGES AND THEMES
%----------------------------------------------------------------------------------------

\documentclass[10pt]{beamer}

\mode<presentation> {

% The Beamer class comes with a number of default slide themes
% which change the colors and layouts of slides. Below this is a list
% of all the themes, uncomment each in turn to see what they look like.

\renewcommand{\familydefault}{\rmdefault}

\graphicspath{ {./figures/} }
\usepackage{graphicx} % Allows including images
\usepackage{booktabs} % Allows the use of \toprule, \midrule and \bottomrule in tables
\usepackage{tikz} % Allows the use of \toprule, \midrule and \bottomrule in tables
\usepackage{hyperref}
\usepackage[printwatermark]{xwatermark}

\usetheme{default}
% \usetheme{AnnArbor}
% \usetheme{Antibes}
% \usetheme{Bergen}
% \usetheme{Berkeley}
% \usetheme{Berlin}
% \usetheme{Boadilla}
% \usetheme{CambridgeUS}
% \usetheme{Copenhagen}
% \usetheme{Darmstadt}
% \usetheme{Dresden}
% \usetheme{Frankfurt}
% \usetheme{Goettingen}
% \usetheme{Hannover}
% \usetheme{Ilmenau}
% \usetheme{JuanLesPins}
% \usetheme{Luebeck}
% \usetheme{Madrid}
% \usetheme{Malmoe}
% \usetheme{Marburg}
% \usetheme{Montpellier}
% \usetheme{PaloAlto}
% \usetheme{Pittsburgh}
% \usetheme{Rochester}
% \usetheme{Singapore}
% \usetheme{Szeged}
% \usetheme{Warsaw}

% As well as themes, the Beamer class has a number of color themes
% for any slide theme. Uncomment each of these in turn to see how it
% changes the colors of your current slide theme.

% \usecolortheme{albatross}
\usecolortheme{beaver}
% \usecolortheme{beetle}
% \usecolortheme{crane}
% \usecolortheme{dolphin}
% \usecolortheme{dove}
% \usecolortheme{fly}
% \usecolortheme{lily}
% \usecolortheme{orchid}
% \usecolortheme{rose}
% \usecolortheme{seagull}
% \usecolortheme{seahorse}
% \usecolortheme{whale}
% \usecolortheme{wolverine}

%\setbeamertemplate{footline} % To remove the footer line in all slides uncomment this line
\setbeamertemplate{footline}[page number] % To replace the footer line in all slides with a simple slide count uncomment this line

\setbeamertemplate{navigation symbols}{} % To remove the navigation symbols from the bottom of all slides uncomment this line

% \setbeamertemplate{background}{
%     \tikz[overlay,remember picture]\node[opacity=0.4]at (current page.center){
%         \includegraphics[width=2cm]{iot_analytics_logo.jpg}
%         }}

}

%----------------------------------------------------------------------------------------
%	TITLE PAGE
%----------------------------------------------------------------------------------------

\title[What Phone do People Like?]{Phone Sentiment Analysis with Web Crawl Data} % The short title appears at the bottom of every slide, the full title is only on the title page

\author{Tuomo Kareoja} % Your name
\institute[Alert! Analytics] % Your institution as it will appear on the bottom of every slide, may be shorthand to save space
{
Alert! Analytics \\ % Your institution for the title page
\medskip
}
\date{\today} % Date, can be changed to a custom date

\begin{document}

\begin{frame}
\titlepage % Print the title page as the first slide
\end{frame}

\begin{frame}
\frametitle{Agenda} % Table of contents slide, comment this block out to remove it
\tableofcontents % Throughout your presentation, if you choose to use \section{} and \subsection{} commands, these will automatically be printed on this slide as an overview of your presentation
\end{frame}

%----------------------------------------------------------------------------------------
%	PRESENTATION SLIDES
%----------------------------------------------------------------------------------------

%------------------------------------------------
\section{Summary of Findings}
%------------------------------------------------

%------------------------------------------------
\subsection{People Like iPhone More}
%------------------------------------------------

\begin{frame}
\frametitle{People Like iPhone More}

Only keeping sites with at least two mentions of the phone




\end{frame}

%------------------------------------------------
\subsection{\ldots but the Training Data for Samsung Galaxy is Sketchy}
%------------------------------------------------

\begin{frame}
\frametitle{\ldots but the Training Data for Samsung Galaxy is Sketchy}

Picture of sentiment and mentions of the phone

\end{frame}

%------------------------------------------------
\subsection{\ldots and the Websites Mentioning iPhone are Weird}
%------------------------------------------------

\begin{frame}
\frametitle{\ldots and the Websites Mentioning iPhone are Weird}

Wordcloud

\end{frame}

%------------------------------------------------
\section{What We Did?}
%------------------------------------------------

%------------------------------------------------
\subsection{Count Instances of Word and Word Combinations from Websites}
%------------------------------------------------

\begin{frame}
\frametitle{Count Instances of Word and Word Combinations from Websites}

Common Crawl
Mentions of phones
Positive and negative overall mentions and specific to certain attributes

\end{frame}

%------------------------------------------------
\subsection{Train Models on Manually Labeled Data and Apply These to the Crawled Data}
%------------------------------------------------

Picture of model comparison

\begin{frame}
\frametitle{Train Models on Manually Labeled Data and Apply These to the Crawled Data}

\end{frame}

%------------------------------------------------
\section{What to Do Better Next Time?}
%------------------------------------------------

%------------------------------------------------
\subsection{Make Sure That the Labeling Makes Sense}
%------------------------------------------------

\begin{frame}
\frametitle{Make Sure That the Labeling Makes Sense}

Was this really done by hand and if so, what were the principals
of evaluation?

If done programmatically, where is the code?

How is it possible that sites that don't contain any mention of the the device can have
emotional sentiment towards it?

\end{frame}

%------------------------------------------------
\subsection{Aggressive Limitations on What Sites to Crawl}
%------------------------------------------------

\begin{frame}
\frametitle{Aggressive Limitations on What Sites to Crawl}

What sites are we trying to find?

Forums are tricky but might be valuable

News are problematic because of clickbait

\end{frame}

%------------------------------------------------
\section{Conclusions}
%------------------------------------------------

\begin{frame}
\frametitle{Conclusions}

\begin{enumerate}
    \item iPhone has a more positive sentiment, but this might because of (fraudulent) giveaways and such
    \item \ldots but if people would want the Galaxy phones as much as iPhone we would expect there to similar
    number of these sites for Samsung Galaxy phones and this does not seem to be the case
    \item More carefully documented labeling needed in the future
    \item Web crawling should be even more targeted
\end{enumerate}

\end{frame}

\begin{frame}
\frametitle{The End}

\LARGE{\centerline{Questions?}}

\end{frame}

%----------------------------------------------------------------------------------------

\end{document}
